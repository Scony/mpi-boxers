\documentclass{article}

\usepackage[polish]{babel}
\usepackage[utf8]{inputenc}
\usepackage[OT4]{fontenc}
\usepackage{graphicx,color}
\usepackage{listings}
\selectlanguage{polish}

\title{\textbf{Laboratorium Podstawy Przetwarzania Rozproszonego}}
\author{Jakub Szwachła \\ 101473 \and  Paweł Lampe \\ 99277}

\begin{document}

\maketitle

\section{Algorytm rozwiązania}
\subsection{Definicja problemu}
treść

\subsection{Założenia przyjętego modelu komunikacji}
\begin{itemize}
\item asynchroniczny system z wymianą komunikatów
\item topologia połączeń: \emph{każdy z każdym / specjalna topologia}
\item wymagana pojemność kanału:  \emph{wiadomości w jednym kierunku / nieskończona}
\item inne wymagane własności sieci komunikacyjnej: \emph{kanały typu FIFO, transmisja rozgłoszeniowa}
\end{itemize}

\subsection{Algorytm wzajemnego wykluczania}
event driven albo tekst

\subsection{Analiza złożoności komunikacyjnej algorytmu}
złożoność pojedynczego przebiegu jednej instancji algorytmu (czyli z punktu widzenia pojedynczego procesu) 
złożoność komunikacyjna pakietowa, wyrażona w liczbie komunikatów
złożoność czasowa przy założeniu jednostkowego czasu przesłania pojedynczego komunikatu w kanale
należy wyznaczyć dokładną złożoność (nie rząd złożoności), a gdy możliwe są różne przypadki – należy podać złożoność pesymistyczną oraz średnią

\section{Implementacja rozwiązania}
lamport.h
\hrule
\lstinputlisting[language=C++]{lamport.h}
\hrule
lamport.cpp
\hrule
\lstinputlisting[language=C++]{lamport.cpp}
\hrule
boxer.h
\hrule
\lstinputlisting[language=C++]{boxer.h}
\hrule
boxer.cpp
\hrule
\lstinputlisting[language=C++]{boxer.cpp}
\hrule

\end{document}
